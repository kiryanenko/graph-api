Этот проект содержит исходные тексты и указания к сборке программ с использованием C++ библиотеки {\ttfamily libspu}.

Основное средство сборки -\/ {\ttfamily cmake}. Чтобы собрать собственный проект выполните\+:


\begin{DoxyCode}
cmake CMakeLists.txt
make
\end{DoxyCode}


Файл {\ttfamily C\+Make\+Lists.\+txt} содержит описатели\+:
\begin{DoxyItemize}
\item {\ttfamily S\+P\+U\+\_\+\+A\+R\+CH} -\/ архитектура СП (32, 64)
\item {\ttfamily S\+P\+U\+\_\+\+S\+I\+M\+U\+L\+A\+T\+OR} -\/ использовать симулятор СП
\end{DoxyItemize}


\begin{DoxyItemize}
\item {\ttfamily \hyperlink{spu_8h_source}{spu.\+h}} -\/ совместимый с C99 заголовочный файл поддержки драйвера СП (не обязателен для включения)
\item {\ttfamily \hyperlink{libspu_8hpp_source}{libspu.\+hpp}} -\/ основной заголовочный файл библиотеки; определяет тип данных data\+\_\+t и основные операции над ним
\item {\ttfamily \hyperlink{structure_8hpp_source}{structure.\+hpp}} -\/ заголовочный файл, описывающий класс структуры СП
\item {\ttfamily \hyperlink{fields_8hpp_source}{fields.\+hpp}} -\/ описание класса Fields разметки типа данных data\+\_\+t на поля
\item {\ttfamily \hyperlink{fields__pointer_8hpp_source}{fields\+\_\+pointer.\+hpp}} -\/ умный указатель на объект класса Fields
\item {\ttfamily \hyperlink{extern__value_8hpp_source}{extern\+\_\+value.\+hpp}} -\/ описание классов, позволяющий хранить большие объекты вне памяти СП
\end{DoxyItemize}


\begin{DoxyEnumerate}
\item Все описания выполнены в пространстве имён S\+PU ({\ttfamily using namespace S\+PU})
\item Структуры в памяти СП представляются как объекты (и управляются объектами)
\item Каждому объекту-\/структуре усваивается уникальный идентификатор -\/ G\+S\+ID (см. тип gsid\+\_\+t)
\item Использование \char`\"{}длинной арифметики\char`\"{} для поддержки любой разрядности регистров СП (разрядная сетка кратна 32, см. тип {\ttfamily data\+\_\+t})
\item Поддержка разметки {\ttfamily data\+\_\+t} на поля с естественным порядком следования, произвольной длиной и любым типом данных имени
\item Поддержка хранения в ОЗУ данных, больших разрядной сетки СП 


\end{DoxyEnumerate}

Типы данных {\ttfamily data\+\_\+t} и {\ttfamily gsid\+\_\+t} представляют собой структуры, в которых заключён массив 32-\/разрядных беззнаковых целых. Над типами данных определены операции в файле {\ttfamily containres\+\_\+operations.\+hpp}. G\+S\+ID является чисто суррогатным ключем структуры в памяти и имеет ограниченную поддержку \char`\"{}длинной арифметики\char`\"{}.

Тип данных {\ttfamily data\+\_\+t} репрезентует данных в регистрах СП и обязан соответствовать им по сути. Тип строго Little-\/endian, при этом кратность элементов 32 разрядам соответствует кратности регистра СП. Компиляция заголовочного файла {\ttfamily \hyperlink{spu_8h_source}{spu.\+h}} с поддержкой С++ позволяет использовать шаблонный конструктор типа {\ttfamily data\+\_\+t} от любого типа.

\#\#\# Пример использования {\ttfamily data\+\_\+t} 
\begin{DoxyCode}
\{C++\}
  data\_t d1;
  data\_t d2 = true;
  data\_t d3 = 10;
  data\_t d4 = 1.100;
  data\_t d5 = 0x1234567890adcdef;

  if(d2)
  \{
    cout << to\_string(d1) << endl;
    cout << to\_string(d2) << endl;
    cout << to\_string(d3) << endl;
    cout << to\_string(d4) << endl;
    cout << to\_string(d5) << endl;
    cout << to\_string(d5 + 1) << endl;
    cout << to\_string(d5 << 8) << endl;
    cout << to\_string(~d5) << endl;
  \}
\end{DoxyCode}
 



Класс {\ttfamily template$<$NameT$>$ class Fields$<$NameT$>$} определят разметку полей данных. Конструктор класса принимает дескриптор длин полей {\ttfamily Fields\+Length$<$NameT$>$}, в котором описывается имя поля с заданным типом NameT и произвольным значением длины.

Конкретные данные могут быть заданы как дескриптором {\ttfamily Fields\+Data$<$NameT$>$}, так и усвоены из-\/вне готовыми данными типа {\ttfamily data\+\_\+t}. Данные по полям доступны с использованием оператора {\ttfamily \mbox{[}Name name\mbox{]}}. Тип преобразуется к {\ttfamily data\+\_\+t} автоматически при необходимости.

По содержимому полей можно итерироваться конструкцией for C++11.

\#\#\# Пример использования {\ttfamily Fields} 
\begin{DoxyCode}
\{C++\}
  Fields<string> F(\{
    \{ "a", 8 \},
    \{ "b", 8 \},
    \{ "c", 8 \},
    \{ "d", 8 \},
  \});

  F = 0x1234;

  cout << to\_string(F["a"]) << endl; // 4
  cout << to\_string(F["b"]) << endl; // 3
  cout << to\_string(F["c"]) << endl; // 2
  cout << to\_string(F["d"]) << endl; // 1

  F = \{
    \{ "a", 255 \},
    \{ "b", 15  \},
    \{ "c", 13  \},
    \{ "d", 0   \},
  \};

  cout << to\_string(F) << endl; // 0x00000000-0x000D0FFF

  for (auto ex : F)
  \{
    cout << ex.name << " is " << to\_string(ex.data) << endl;
  \}
\end{DoxyCode}


Для класса {\ttfamily Fields} существует \char`\"{}умный\char`\"{} указатель -\/ класс {\ttfamily template$<$NameT$>$ class Fields\+Pointer$<$NameT$>$}, который конструируется либо от указателя на {\ttfamily Fields}, либо от объекта такого же класса ({\ttfamily Fields\+Pointer}). Класс призван эффективно управлять указателем на разметку полей и не допускать повторного удаления. Этот класс появился поскольку требуется обеспечить множество указателей на класс {\ttfamily Fields} с прозрачным синтаксисом и удалением только одного (без сегментации памяти). Синтаксис этого класс прозрачен как для {\ttfamily Fields}.

Для того, чтобы классы {\ttfamily Fields} и {\ttfamily Fields\+Pointer} имели абсолютно одинаковый синтаксис, создан абстрактный класс {\ttfamily template$<$NameT$>$ class Abstract\+Fields$<$NameT$>$}. Он содержит описания всех методов, что применяются при работе с разметкой полей. 



Структуры СП представляются объектами класса {\ttfamily Structure}. Обобщённый интерфейс работы со структурой описывается абстрактным классом Abstract\+Structure (см. {\ttfamily \hyperlink{abstract__structure_8hpp_source}{libspu/abstract\+\_\+structure.\+hpp}}). Структура поддерживает следующие команды СП -\/ методы\+:


\begin{DoxyItemize}
\item status\+\_\+t insert(key, value) -\/ Добавить новую пару key-\/value
\item insert\+Vector(insert\+\_\+vector) -\/ Добавить в структуру вектор значений
\item status\+\_\+t del(key) -\/ Удалить пару key-\/value
\item pair\+\_\+t search(key) -\/ Поиск ключа
\item pair\+\_\+t min() -\/ Поиск минимальной по ключу пары
\item pair\+\_\+t max() -\/ Поиск максимальной по ключу пары
\item pair\+\_\+t next(key) -\/ Следующая по ключу пара
\item pair\+\_\+t prev(key) -\/ Предыдущая по ключу пара
\item pair\+\_\+t nsm(key) -\/ Следующая соседняя по ключу снизу пара
\item pair\+\_\+t ngr(key) -\/ Следующая соседняя по ключу сверху пара
\item u32 get\+\_\+power() -\/ Получить мощность структуры
\item gsid\+\_\+t get\+\_\+gsid() -\/ получить G\+S\+ID структуры
\end{DoxyItemize}

\#\#\# Хранилище пары ключ-\/значение {\ttfamily pair\+\_\+t} имеет поля ключа, значения и статуса выполнения операции СП 
\begin{DoxyCode}
\{C++\}
struct pair\_t
  \{
    key\_t    key;
    value\_t  value;
    status\_t status;
  \};
\end{DoxyCode}


Каждая операция имеет флаги выполнения, вызовы по-\/умолчанию исполняются флаги оптимизированные для этой команды. Возможные флаги\+:


\begin{DoxyItemize}
\item N\+O\+\_\+\+F\+L\+A\+GS -\/ Нет флагов исполнения
\item P\+\_\+\+F\+L\+AG -\/ Флаг ожидания окончания исполнения операции и возврата результата
\item Q\+\_\+\+F\+L\+AG -\/ Флаг помещения операции в очередь исполнения
\item R\+\_\+\+F\+L\+AG -\/ Флага сброса очереди исполнения операция 


\end{DoxyItemize}

Структура описывается классом {\ttfamily template$<$Key\+NameT,Value\+NameT$>$ class Structure}. Специализированный класс {\ttfamily Structure$<$void, void$>$} реализует описанный выше интерфейс {\ttfamily Abstract\+Structure}.

Полная шаблонная реализация включает в себя разметку полей ключа и значения. Для описания ключа и значения можно использовать дескрипторы данных полей. Можно получить указатели на объекты классов разметки полей (класс {\ttfamily Fields\+Pointer}) методами {\ttfamily key()} и {\ttfamily value()}. Полученные таким образом ключи и значения будут соответствовать последним данным, полученным из СП или переданному туда (для операций поиска в случае успешного завершения).

Библиотека описывает функции преобразования данных к классу строки C++ {\ttfamily std\+::string}. Эти функции описаны для типа данных {\ttfamily data\+\_\+t}, статуса завершения операции СП и контейнера пары значений ключ-\/значения СП {\ttfamily pair\+\_\+t}. Преобразование типа {\ttfamily pair\+\_\+t} к строке показывает его в формате {\ttfamily Статус завершения\+: Ключ \+: Значение}.

\#\#\# Пример управления специализированной структурой СП 
\begin{DoxyCode}
\{C++\}
  Structure<> S;
  S.insert(1, 1);
  S.insertVector(\{
    \{2, 2\},
    \{3, 3\},
    \{4, 4\},
    \{5, 5\},
  \});
  cout << to\_string(S.min()) << endl;     // OK: 0x00000000-0x00000001 : 0x00000000-0x00000001
  cout << to\_string(S.max()) << endl;     // OK: 0x00000000-0x00000005 : 0x00000000-0x00000005
  cout << to\_string(S.search(4)) << endl; // OK: 0x00000000-0x00000004 : 0x00000000-0x00000004
\end{DoxyCode}


\#\#\# Пример управления структурой СП с разметкой полей 
\begin{DoxyCode}
\{C++\}
  /* Разметка ключа */
  Structure<string> S\_k(\{ // Задание старшинства разрядов естественная: снизу-вверх => слева-направо
    \{ "k\_a", 8 \},
    \{ "k\_b", 8 \},
  \});

  /* Разметка значения */
  Structure<void, string> S\_v(\{
    \{ "v\_a", 8 \},
    \{ "v\_b", 8 \},
  \});

  /* Разметка и ключа и значения */
  Structure<string, string> S\_kv(\{
    \{ "k\_a", 8 \},
    \{ "k\_b", 8 \},
  \},\{
    \{ "v\_a", 8 \},
    \{ "v\_b", 8 \},
  \});

  /* Получение указателей на объекты ключа и значения */
  auto S\_kv\_key   = S\_kv.key();  // Автоматически получаются правильные шаблоны класса FieldsPointer
  auto S\_kv\_value = S\_kv.value();

  S\_kv.insert(\{
    \{"k\_b", 255\},
  \},\{
    \{"v\_a", 16\},
  \});

  cout << to\_string(S\_kv.min()) << endl;                               // OK: 0x00000000-0x0000ff00 :
       0x00000000-0x00000010
  cout << to\_string(S\_kv\_key) << " " << to\_string(S\_kv\_value) << endl; // 0x00000000-0x0000ff00
       0x00000000-0x00000010
\end{DoxyCode}
 



Библиотека имеет класс {\ttfamily Extern\+Value} для хранения больших значений. Значение помещается в список, а её уникальный идентификатор -\/ в память СП.


\begin{DoxyCode}
\{C++\}
  Structure<> S;
  pair\_t pair;

  /* Сохранение */
  string str = "This string stored at hash map. In SPU stored id for a string";
  BaseExternValue extern\_val = HashMapExternValue<string>(str);
  S.insert(1, extern\_val);

  /* Поиск */
  pair = S.search(1);
  string res\_str = (HashMapExternValue<string>) pair.value;
  cout << res\_str << endl;

  /* Произвольная не пакетированная структура */
  struct Point \{double x; double y; double z;\};
  Point p = \{1.5, 2.3, 3.7\};
  HashMapExternValue<Point> point\_ext;
  point\_ext << p; // Операторы << и >> делают тоже, что и методы set и get
  S.insert(2, point\_ext);
  pair = S.search(2);
  if (pair.status == OK) \{
      point\_ext << pair;
      point\_ext >> p;
      cout << "Point struct X=" << p.x << " Y=" << p.y << " Z=" << p.z << endl;
  \}
\end{DoxyCode}
 



Управление СП осуществляется классом {\ttfamily Base\+Structure}. При помощи класса {\ttfamily Fileops} он передаёт команды СП и получает результат. Здесь реализованы непосредственные передачи от символьного устройства СП к библиотеке и запись данных.

Альтернативно при установке макроопределения {\ttfamily S\+P\+U\+\_\+\+S\+I\+M\+U\+L\+A\+T\+OR} место класса {\ttfamily Base\+Structure} занимает класс {\ttfamily Simulator}. Этот класс на основе {\ttfamily std\+::map} симулирует (эмулирует) поведение СП по всем поддерживаемым операциям.

Диаграмма классов библиотеки структур приведена ниже.

 

\hypertarget{md_spu-api__readme_autotoc_md32}{}\section{Сложный пример алгоритма с использованием библиотеки -\/ Алгоритм Дейкстры}\label{md_spu-api__readme_autotoc_md32}

\begin{DoxyCode}
\{C++\}
#include <iostream>

#include <libspu.hpp>
#include <structure.hpp>

using namespace std;
using namespace SPU;

#define INF    0xf
#define u\_cnt  5 

/* Graph representation

       7
 '1' ------ '3' 
  |        /  \(\backslash\) 7
  |      /      \(\backslash\) 
 2|   4 /       '5'
  |   /         / 
  | /   1     / 6
 '2' ------ '4'

*/

/*************************************
  Structures definitions
*************************************/

/* Graph of convergence G */
Structure<void, string> G(\{ // Has data fields but not key fields
  \{ "Adj[u]", 16 \}, // Has own Fields
  \{ "w[u]",   20 \}, // Also has own Fields
  \{ "d[u]",   4  \}, // Max distance is 15
  \{ "p[u]",   4  \},
  \{ "u∈Q",    1  \}, // Boolean
\});
auto G\_value = G.value();

/* Adj[u] fields */
Fields<> Adj\_u(\{ // Every field is boolean
  \{ 1, 1 \},
  \{ 2, 1 \},
  \{ 3, 1 \},
  \{ 4, 1 \},
  \{ 5, 1 \},
\});

/* w[u] fields */
Fields<> w\_u(\{ // Max distance is 16 
  \{ 1, 4 \},
  \{ 2, 4 \},
  \{ 3, 4 \},
  \{ 4, 4 \},
  \{ 5, 4 \},
\});

/* Structure of consideration Q */
Structure<string> Q(\{
  \{ "u",    8 \},
  \{ "d[u]", 4 \}, // d[u] is more important
\});
auto Q\_key = Q.key(); // Get Fields to separate Q key

/*************************************
  End of structures definitions
*************************************/

/* Helpers */
void G\_init();
void Q\_init();
void G\_print();
void Q\_print();

int main()
\{
  cout << "Starting Dijkstra algorithm" << endl;

  /* G */
  G\_init();
  G\_print();

  /* Q */
  Q\_init();
  Q\_print();

  cout << "Starting" << endl;

  /*************************************
    Main algorithm
  *************************************/

  while(Q.get\_power())
  \{
    /* Get first node from Q and delete it */
    Q.min();
    u8 u = Q\_key["u"]; // Index of node
    Q.del(Q\_key);

    /* Get G\_value from Q's "u" */
    G.search(u);
    Adj\_u      = G\_value["Adj[u]"];
    data\_t d\_u = G\_value["d[u]"];

    /* Unset u∈Q */
    G\_value["u∈Q"] = false;
    data\_t u\_value = G\_value;

    /* Check out all v's from Adj[u] */
    for(auto ex : Adj\_u)
    \{
      /* If v in Adj[u] */
      if(ex.data)
      \{
        /* Search for v */
        u8 v = ex.name;
        G.search(v); // Now G\_value is value of v key
        w\_u = G\_value["w[u]"];

        /* v is in Q */
        if(G\_value["u∈Q"])
        \{
          /* Delete v from Q */
          Q\_key["u"]    = v;
          Q\_key["d[u]"] = G\_value["d[u]"];
          Q.del(Q\_key);

          /* Create new length statement */
          data\_t len = d\_u + w\_u[u];
          if( G\_value["d[u]"] > len )
          \{
            /* Set new data */
            G\_value["d[u]"] = len;
            G\_value["p[u]"] = u;

            /* Save v state */
            G.insert(v, G\_value);

            /* Insert new v in Q */
            Q\_key["d[u]"] = G\_value["d[u]"];
          \}

          Q.insert(Q\_key, 0);
        \}
      \}
    \}

    /* Save u state */
    G.insert(u, u\_value);

    G\_print();
    Q\_print();

    if (Q.get\_power())
    \{
      cout << "Turn" << endl;
    \}
  \}

  cout << "Ended" << endl;

  return 0;
\}

/*************************************
  G initialization
*************************************/
void G\_init()
\{
  /* u = 1 */
  Adj\_u  = \{ \{ 2, true \}, \{ 3, true \} \};
  w\_u    = \{ \{ 2, 2    \}, \{ 3, 7    \} \};
  G.insert(1, \{
    \{ "Adj[u]", Adj\_u \},
    \{ "w[u]",   w\_u   \},
    \{ "d[u]",   0     \},
    \{ "p[u]",   0     \},
    \{ "u∈Q",    true  \} 
  \});

  /* u = 2 */
  Adj\_u  = \{ \{ 1, true \}, \{ 3, true \}, \{ 4, true \} \};
  w\_u    = \{ \{ 1, 2    \}, \{ 3, 4    \}, \{ 4, 1    \} \};
  G.insert(2, \{
    \{ "Adj[u]", Adj\_u \},
    \{ "w[u]",   w\_u   \},
    \{ "d[u]",   INF   \},
    \{ "p[u]",   0     \},
    \{ "u∈Q",    true  \} 
  \});

  /* u = 3 */
  Adj\_u  = \{ \{ 1, true \}, \{ 2, true \}, \{ 4, true \}, \{ 5, true \} \};
  w\_u    = \{ \{ 1, 7    \}, \{ 2, 4    \}, \{ 4, 2    \}, \{ 5, 7    \} \};
  G.insert(3, \{
    \{ "Adj[u]", Adj\_u \},
    \{ "w[u]",   w\_u   \},
    \{ "d[u]",   INF   \},
    \{ "p[u]",   0     \},
    \{ "u∈Q",    true  \} 
  \});

  /* u = 4 */
  Adj\_u  = \{ \{ 2, true \}, \{ 3, true \}, \{ 5, true \} \};
  w\_u    = \{ \{ 2, 1    \}, \{ 3, 2    \}, \{ 5, 6    \} \};
  G.insert(4, \{
    \{ "Adj[u]", Adj\_u \},
    \{ "w[u]",   w\_u   \},
    \{ "d[u]",   INF   \},
    \{ "p[u]",   0     \},
    \{ "u∈Q",    true  \} 
  \});

  /* u = 5 */
  Adj\_u  = \{ \{ 3, true \}, \{ 4, true \} \};
  w\_u    = \{ \{ 3, 7    \}, \{ 4, 6    \} \};
  G.insert(5, \{
    \{ "Adj[u]", Adj\_u \},
    \{ "w[u]",   w\_u   \},
    \{ "d[u]",   INF   \},
    \{ "p[u]",   0     \},
    \{ "u∈Q",    true  \} 
  \});
\}

/*************************************
  Q initialization
*************************************/
void Q\_init()
\{
  /* Q first node init */
  Q.insert(
    \{ \{ "d[u]", 0 \},  \{ "u", 1 \} \},
    0
  );

  /* Q other nodes init */
  for(u8 u=2; u<=u\_cnt; u++)
  \{
    Q.insert(
      \{ \{ "d[u]", INF \},  \{ "u", u \} \},
      0
    );
  \}
\}

/*************************************
  G printing
*************************************/
void G\_print()
\{
  /* Print out */
  cout << "G graph is:" << endl;
  for(u8 u=1; u<=u\_cnt; u++)
  \{
    pair\_t pair =  G.search(u);
    cout << "\(\backslash\)t u = " << to\_string(pair.key) <<
      ":  " << to\_string(pair.value, true) << endl;
  \}

  cout << endl;
\}

/*************************************
  Q printing
*************************************/
void Q\_print()
\{
  if (Q.get\_power())
  \{
    /* Print out */
    cout << "Q structures keys are:" << endl;

    /* First node */
    pair\_t pair = Q.min();
    cout << "\(\backslash\)t " << to\_string(pair.key, true) << endl;

    /* Other nodes */
    for(u8 u=2; u<=Q.get\_power(); u++)
    \{
      pair = Q.next(pair.key);
      cout << "\(\backslash\)t " << to\_string(pair.key, true) << endl;
    \}

    cout << endl;
  \}
  else
  \{
    cout << "Q is empty" << endl;
  \}
\}
\end{DoxyCode}
 