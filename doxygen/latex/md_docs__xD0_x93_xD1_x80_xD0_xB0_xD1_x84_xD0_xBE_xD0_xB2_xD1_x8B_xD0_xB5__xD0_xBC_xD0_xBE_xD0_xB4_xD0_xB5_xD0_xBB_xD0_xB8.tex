\hypertarget{md_docs__xD0_x93_xD1_x80_xD0_xB0_xD1_x84_xD0_xBE_xD0_xB2_xD1_x8B_xD0_xB5__xD0_xBC_xD0_xBE_xD0_xB4_xD0_xB5_xD0_xBB_xD0_xB8_autotoc_md0}{}\section{ВВЕДЕНИЕ}\label{md_docs__xD0_x93_xD1_x80_xD0_xB0_xD1_x84_xD0_xBE_xD0_xB2_xD1_x8B_xD0_xB5__xD0_xBC_xD0_xBE_xD0_xB4_xD0_xB5_xD0_xBB_xD0_xB8_autotoc_md0}
В качестве основного обрабатывающего блока современных ЭВМ выступает арифметико-\/логическое устройство. Вместе с тем, при решении практических задач используется существенно большее количество математических операций, включая операции над множествами в дискретной математике.

В МГТУ им. Н.\+Э. Баумана проведен полный цикл создания принципиально новой универсальной вычислительной системы, начиная от создания принципов и моделей и заканчивая созданием опытного образца, проведения тестов и испытаний \mbox{[}2\mbox{]}. Разработано принципиально новое вычислительное устройство\+: Процессор с набором команд дискретной математики (Процессор обработки структур, далее СП), реализующее набор команд дискретной математики высокого уровня над множествами и структурами данных. Новая архитектура позволяет более эффективно решать задачи дискретной оптимизации, основанные на моделях множеств, графов, и отношений.

В данной работе будет произведена разработка моделей представления различных типов графов для систем с набором дискретной математики.\hypertarget{md_docs__xD0_x93_xD1_x80_xD0_xB0_xD1_x84_xD0_xBE_xD0_xB2_xD1_x8B_xD0_xB5__xD0_xBC_xD0_xBE_xD0_xB4_xD0_xB5_xD0_xBB_xD0_xB8_autotoc_md1}{}\section{Набор команд дискретном математики}\label{md_docs__xD0_x93_xD1_x80_xD0_xB0_xD1_x84_xD0_xBE_xD0_xB2_xD1_x8B_xD0_xB5__xD0_xBC_xD0_xBE_xD0_xB4_xD0_xB5_xD0_xBB_xD0_xB8_autotoc_md1}
Микропроцессор Leonhard x64 хранит информацию о множествах в виде неперекрывающихся B+ деревьев. Последняя версия набора команд Leonhard x64 была расширена двумя новыми инструкциями (N\+SM и N\+GR) для обеспечения требований некоторых алгоритмов. Каждая инструкция набора включает до трех операндов (таблица 1)\+:

Таблица 1 -\/ Формат данных Leonhard x64

\tabulinesep=1mm
\begin{longtabu} spread 0pt [c]{*{3}{|X[-1]}|}
\hline
\rowcolor{\tableheadbgcolor}\textbf{ Структура }&\textbf{ Ключ }&\textbf{ Значение  }\\\cline{1-3}
\endfirsthead
\hline
\endfoot
\hline
\rowcolor{\tableheadbgcolor}\textbf{ Структура }&\textbf{ Ключ }&\textbf{ Значение  }\\\cline{1-3}
\endhead
3 бита &64 бита &64 бита \\\cline{1-3}
\end{longtabu}


Набор команд состоит из 20 высокоуровневых кодов операций, перечисленных ниже.
\begin{DoxyItemize}
\item {\bfseries Search (S\+R\+CH)} выполняет поиск значения, связанного с ключом.
\item {\bfseries Insert (I\+NS)} вставляет пару ключ-\/значение в структуру. S\+PU обновляет значение, если указанный ключ уже находится в структуре.
\item {\bfseries Операция Delete (D\+EL)} выполняет поиск указанного ключа и удаляет его из структуры данных.
\item {\bfseries Neighbors (N\+SM, N\+GR)} выполняют поиск соседнего ключа, который меньше (или больше) заданного и возвращает его значение. Операции могут быть использованы для эвристических вычислений, где интерполяция данных используется вместо точных вычислений (например, кластеризация или агрегация).
\item {\bfseries Maximum /minimum (M\+AX, M\+IN)} ищут первый или последний ключи в структуре данных.
\item {\bfseries Операция Cardinality (C\+NT)} определяет количество ключей, хранящихся в структуре.
\item {\bfseries Команды A\+ND, OR, N\+OT} выполняют объединения, пересечения и дополнения в двух структурах данных.
\item {\bfseries Срезы (LS, GR, L\+S\+EQ, G\+R\+EQ)} извлекают подмножество одной структуры данных в другую.
\item {\bfseries Переход к следующему или предыдущему (N\+E\+XT, P\+R\+EV)} находят соседний (следующий или предыдущий) ключ в структуре данных относительно переданного ключа. В связи с тем, что исходный ключ должен обязательно присутствовать в структуре данных, операции N\+E\+X\+T/\+P\+R\+EV отличаются от N\+S\+M/\+N\+GR.
\item {\bfseries Удаление структуры (D\+E\+LS)} очищает все ресурсы, используемые заданной структурой.
\item {\bfseries Команда Squeeze (SQ)} дефрагментирует блоки памяти D\+SM, используемые структурой.
\item {\bfseries Команда Jump (JT)} указывает S\+PU код ветвления, который должен быть синхронизирован с C\+PU (команда доступна только в режиме M\+I\+SD).
\end{DoxyItemize}\hypertarget{md_docs__xD0_x93_xD1_x80_xD0_xB0_xD1_x84_xD0_xBE_xD0_xB2_xD1_x8B_xD0_xB5__xD0_xBC_xD0_xBE_xD0_xB4_xD0_xB5_xD0_xBB_xD0_xB8_autotoc_md2}{}\section{Анализ требований, предъявляемых к графовым моделям для систем с набором дискретной математики}\label{md_docs__xD0_x93_xD1_x80_xD0_xB0_xD1_x84_xD0_xBE_xD0_xB2_xD1_x8B_xD0_xB5__xD0_xBC_xD0_xBE_xD0_xB4_xD0_xB5_xD0_xBB_xD0_xB8_autotoc_md2}
Самое главное требование, предъявляемое к моделям является, что графы, построенные на базе этой модели, должны соответствовать всем свойствам данного типа графа. Также должен осуществляться эффективный поиск смежных вершин. Если граф взвешанный, то по модели должен осуществляться поиск минимальных и максимальных ребер.

Для осуществления эффективного поиска, важно правильно задавать порядок аргументов в ключе. Наиболее приорететные аргумнты для поиска должны находится в старших разрядах ключа.\hypertarget{md_docs__xD0_x93_xD1_x80_xD0_xB0_xD1_x84_xD0_xBE_xD0_xB2_xD1_x8B_xD0_xB5__xD0_xBC_xD0_xBE_xD0_xB4_xD0_xB5_xD0_xBB_xD0_xB8_autotoc_md3}{}\section{Графовые модели для системы с набором дискретной математики}\label{md_docs__xD0_x93_xD1_x80_xD0_xB0_xD1_x84_xD0_xBE_xD0_xB2_xD1_x8B_xD0_xB5__xD0_xBC_xD0_xBE_xD0_xB4_xD0_xB5_xD0_xBB_xD0_xB8_autotoc_md3}
\hypertarget{md_docs__xD0_x93_xD1_x80_xD0_xB0_xD1_x84_xD0_xBE_xD0_xB2_xD1_x8B_xD0_xB5__xD0_xBC_xD0_xBE_xD0_xB4_xD0_xB5_xD0_xBB_xD0_xB8_autotoc_md4}{}\subsection{1 Модель для ориентированного остовного графа}\label{md_docs__xD0_x93_xD1_x80_xD0_xB0_xD1_x84_xD0_xBE_xD0_xB2_xD1_x8B_xD0_xB5__xD0_xBC_xD0_xBE_xD0_xB4_xD0_xB5_xD0_xBB_xD0_xB8_autotoc_md4}
Данная модель подходит для невзваешанных ориентированных остовных графов и обычных ориентированных графов, у которых нельзя двумя и более ребрами соединить две одинаковых вершины. Модель использует всего одну S\+PU структуру\+:


\begin{DoxyCode}
| Ключ                            | Значение       |
| ------------------------------- | -------------- |
| id вершины | id смежной вершины |                |
| -----------| -------------------| -------------- |
| id вершины | 0..0               | Данные вершины | 
\end{DoxyCode}


Структура хранит смежные вершины. Старшим аргументом ключа является идентификатор вершины, а младшим -\/ идентификатор смежной вершины. Данные вершины хранится в значении для записи с ключом, у которого аргумент {\ttfamily id вершины} соответствует этой вершине и аргумент {\ttfamily id смежной вершины} равен 0.

Для неориентированного остовного графа необходимо каждое ребро представлять дважды относительно входящих в него вершин.

{\bfseries Ограничения\+:}
\begin{DoxyEnumerate}
\item Идентификаторы вершин не должны быть равны 0.
\item Нельзя двумя и более ребрами соединить две одинаковых вершины.
\end{DoxyEnumerate}\hypertarget{md_docs__xD0_x93_xD1_x80_xD0_xB0_xD1_x84_xD0_xBE_xD0_xB2_xD1_x8B_xD0_xB5__xD0_xBC_xD0_xBE_xD0_xB4_xD0_xB5_xD0_xBB_xD0_xB8_autotoc_md5}{}\subsection{2 Модель для взвешанного ориентированного остовного графа}\label{md_docs__xD0_x93_xD1_x80_xD0_xB0_xD1_x84_xD0_xBE_xD0_xB2_xD1_x8B_xD0_xB5__xD0_xBC_xD0_xBE_xD0_xB4_xD0_xB5_xD0_xBB_xD0_xB8_autotoc_md5}
Данная модель подходит для взвешанных ориентированных остовных графов. Модель использует всего одну S\+PU структуру\+:


\begin{DoxyCode}
| Ключ                                   | Значение       |
| ---------- + ---- + ------------------ + -------------- |
| id вершины | Вес  | id смежной вершины |                |
| ---------- + ---- + ------------------ + -------------- |
| id вершины | 0..0                      | Данные вершины |
| id вершины | 0..0 | id смежной вершины | Вес ребра      | 
\end{DoxyCode}


Структура хранит смежные вершины. Старшим аргументом ключа является идентификатор вершины, вторым -\/ вес ребра, а младшим -\/ идентификатор смежной вершины. Данные вершины хранятся в значении для записи с ключом, у которого аргумент {\ttfamily id вершины} соответствует этой вершине и аргумент {\ttfamily id смежной вершины} равен 0.

Для нахождения веса ребра между вершинами необходимо добавлять запись, в которой аргументы {\ttfamily id вершины} и {\ttfamily id смежной вершины} соответствуют смежным вершинам, аргумент {\ttfamily Вес} равен 0, а значение равно весу ребра.

Для неориентированного остовного графа необходимо каждое ребро представлять дважды относительно входящих в него вершин.

Отличие данной модели от предыдущей -\/ в S\+PU структуру был добавлен аргумент {\ttfamily Вес ребра}, благодаря которому будет осуществляться быстрый поиск смежной вершины, соединенной через ребро с минимальным или максимальным весом.

{\bfseries Ограничения\+:}
\begin{DoxyEnumerate}
\item Идентификаторы вершин не должны быть равны 0.
\item Вес ребра представлен натуральным числом.
\end{DoxyEnumerate}\hypertarget{md_docs__xD0_x93_xD1_x80_xD0_xB0_xD1_x84_xD0_xBE_xD0_xB2_xD1_x8B_xD0_xB5__xD0_xBC_xD0_xBE_xD0_xB4_xD0_xB5_xD0_xBB_xD0_xB8_autotoc_md6}{}\subsection{3 Модель для ориентированного графа}\label{md_docs__xD0_x93_xD1_x80_xD0_xB0_xD1_x84_xD0_xBE_xD0_xB2_xD1_x8B_xD0_xB5__xD0_xBC_xD0_xBE_xD0_xB4_xD0_xB5_xD0_xBB_xD0_xB8_autotoc_md6}
Модель подходит для ориентированных графов и использует одну структуру S\+PU\+:


\begin{DoxyCode}
| Ключ                                         | Значение             |
| -------------------------------------------- | -------------------- |
| id вершины | id ребра | id смежной вершины   |                      |
| -----------| ------------------------------- | -------------------- |
| 0..0       | id ребра | id смежной вершины 1 | id смежной вершины 2 | 
| id вершины | 0..0                            | Данные вершины       | 
\end{DoxyCode}


Структура хранит смежные вершины. Старшим аргументом ключа является идентификатор вершины, вторым -\/ идентификотор ребра, а младшим -\/ идентификатор смежной вершины. Данные вершины хранятся в значении для записи с ключом, у которого аргумент {\ttfamily id вершины} соответствует этой вершине, а остальные аргументы равны 0.

Для поиска по идентификатору вершины необходимо добавлять запись, в которой аргумент {\ttfamily id вершины} равен 0, {\ttfamily id ребра} соответствует ребру, а младший аргумент и значение соответствует идентификаторам смежных вершин. Поиск такой записи будет происходить с помощью команды {\ttfamily ngr(0, id, 0)}.

Для неориентированного графа необходимо каждое ребро представлять дважды относительно входящих в него вершин.

{\bfseries Ограничения\+:} Идентификаторы вершин не должны быть равны 0.\hypertarget{md_docs__xD0_x93_xD1_x80_xD0_xB0_xD1_x84_xD0_xBE_xD0_xB2_xD1_x8B_xD0_xB5__xD0_xBC_xD0_xBE_xD0_xB4_xD0_xB5_xD0_xBB_xD0_xB8_autotoc_md7}{}\subsection{4 Модель для взвешанного ориентированного графа}\label{md_docs__xD0_x93_xD1_x80_xD0_xB0_xD1_x84_xD0_xBE_xD0_xB2_xD1_x8B_xD0_xB5__xD0_xBC_xD0_xBE_xD0_xB4_xD0_xB5_xD0_xBB_xD0_xB8_autotoc_md7}
Модель подходит для взваешанных ориентированных графов и использует одну структуру S\+PU\+:


\begin{DoxyCode}
| Ключ                                               | Значение             |
| ---------- + --- + -------- + -------------------- + -------------------- |
| id вершины | Вес | id ребра | id смежной вершины   |                      |
| ---------- + --- + -------- + -------------------- + -------------------- |
| 0..0             | id ребра | id смежной вершины 1 | id смежной вершины 2 | 
| id вершины | 0..0                                  | Данные вершины       | 
\end{DoxyCode}


Структура хранит смежные вершины. Старший аргумент ключа является идентификатор вершины, второй -\/ вес ребра, третий -\/ идентификотор ребра, а младший -\/ идентификатор смежной вершины. Данные вершины хранится в значении для записи с ключом, у которого аргумент {\ttfamily id вершины} соответствует этой вершине, а остальные аргументы равны 0.

Отличие данной модели от предыдущей -\/ в S\+PU структуру был добавлен аргумент {\ttfamily Вес ребра}, благодаря которому, будет осуществлятся быстрый поиск смежной вершины, соединенной ребом с минимальным / максимальным весом.

Для поиска по идентификатору вершины необходимо добавлять запись, в которой аргумент {\ttfamily id вершины} и {\ttfamily Вес} равны 0, {\ttfamily id ребра} соответствует ребру, а младший аргумент и значение соответствует идентификаторам смежных вершин. Поиск такой записи будет происходить с помощью команды {\ttfamily ngr(0, 0, id, 0)}.

{\bfseries Ограничения\+:}
\begin{DoxyEnumerate}
\item Идентификаторы вершин не должны быть равны 0.
\item Вес ребра представлен целым положительным числом.
\end{DoxyEnumerate}\hypertarget{md_docs__xD0_x93_xD1_x80_xD0_xB0_xD1_x84_xD0_xBE_xD0_xB2_xD1_x8B_xD0_xB5__xD0_xBC_xD0_xBE_xD0_xB4_xD0_xB5_xD0_xBB_xD0_xB8_autotoc_md8}{}\subsection{5 Модель для гиперграфа}\label{md_docs__xD0_x93_xD1_x80_xD0_xB0_xD1_x84_xD0_xBE_xD0_xB2_xD1_x8B_xD0_xB5__xD0_xBC_xD0_xBE_xD0_xB4_xD0_xB5_xD0_xBB_xD0_xB8_autotoc_md8}
Данная модель подходит для гиперграфов и использует 2 структуры S\+PU\+:


\begin{DoxyEnumerate}
\item {\bfseries Связь вершина -\/$>$ ребро.}

``` $\vert$ Ключ $\vert$ Значение $\vert$ $\vert$ -\/-\/-\/-\/-\/-\/-\/-\/-\/-\/-\/-\/-\/-\/-\/-\/-\/-\/--- $\vert$ -\/-\/-\/-\/-\/-\/-\/-\/-\/-\/-\/--- $\vert$ \tabulinesep=1mm
\begin{longtabu} spread 0pt [c]{*{3}{|X[-1]}|}
\hline
\rowcolor{\tableheadbgcolor}\textbf{ id вершины }&\textbf{ id ребра }&\textbf{ }\\\cline{1-3}
\endfirsthead
\hline
\endfoot
\hline
\rowcolor{\tableheadbgcolor}\textbf{ id вершины }&\textbf{ id ребра }&\textbf{ }\\\cline{1-3}
\endhead
id вершины &0..0 &Данные вершины \\\cline{1-3}
\end{longtabu}
```

Структура хранит вершины и исходящие из них ребра. По этой структуре происходит поиск смежных ребер для указанной вершины. Старший аргумент ключа является идентификатор вершины, а младший -\/ идентификатор смежного ребра. Данные вершины хранится в значении для записи с ключом, у которого аргумент {\ttfamily id вершины} соответствует этой вершине и аргумент {\ttfamily id ребра} равен 0.
\item {\bfseries Связь ребро -\/$>$ вершина.}

``` $\vert$ Ключ $\vert$ Значение $\vert$ $\vert$ -\/-\/-\/-\/-\/-\/-\/-\/-\/-\/-\/-\/-\/-\/-\/-\/-\/-\/--- $\vert$ -\/-\/-\/-\/-\/-\/--- $\vert$ \tabulinesep=1mm
\begin{longtabu} spread 0pt [c]{*{3}{|X[-1]}|}
\hline
\rowcolor{\tableheadbgcolor}\textbf{ id ребра }&\textbf{ id вершины }&\textbf{ }\\\cline{1-3}
\endfirsthead
\hline
\endfoot
\hline
\rowcolor{\tableheadbgcolor}\textbf{ id ребра }&\textbf{ id вершины }&\textbf{ }\\\cline{1-3}
\endhead
id ребра &0..0 &Вес ребра \\\cline{1-3}
\end{longtabu}
```

Структура хранит ребра и вершины, в которые входят эти ребра. По этой структуре происходит поиск смежных вершин для указанного ребра. Старший аргумент ключа является идентификатор ребра, а младший -\/ идентификатор смежной вершины.

Для взвешанных графов в значении для записи с ключом, у которого аргумент {\ttfamily id ребра} соответствует указанному ребру, а аргумент {\ttfamily id вершины} равен 0, значение будет соответствовать весу ребра. Таким образом, модель будет обеспечивать быстрое изменение веса ребра, однако, чтобы получить ребро с минимальны весом, придется вытащить все ребра и найти среди них ребро с минимальным весом.
\end{DoxyEnumerate}

Может показаться, что если в первой структуре размещать исходящие из вершины ребра, а во второй -\/ входящие в вершину ребра, то модель будет подходить для ультрографов. Однако, в таком случае нельзя по вершине будет найти все входящие в неё ребра.

{\bfseries Ограничения\+:} Идентификаторы вершины и ребра не должны быть равны 0.\hypertarget{md_docs__xD0_x93_xD1_x80_xD0_xB0_xD1_x84_xD0_xBE_xD0_xB2_xD1_x8B_xD0_xB5__xD0_xBC_xD0_xBE_xD0_xB4_xD0_xB5_xD0_xBB_xD0_xB8_autotoc_md9}{}\subsection{6 Модель для ультраграфа}\label{md_docs__xD0_x93_xD1_x80_xD0_xB0_xD1_x84_xD0_xBE_xD0_xB2_xD1_x8B_xD0_xB5__xD0_xBC_xD0_xBE_xD0_xB4_xD0_xB5_xD0_xBB_xD0_xB8_autotoc_md9}
Данная модель подходит для ультрарафов и использует 2 структуры S\+PU\+:


\begin{DoxyEnumerate}
\item {\bfseries Связь вершина -\/$>$ ребро.}

``` $\vert$ Ключ $\vert$ Значение $\vert$ $\vert$ -\/-\/-\/-\/-\/-\/-\/-\/-\/-\/-\/-\/-\/--- $\vert$ -\/-\/-\/-\/-\/-\/-\/--- $\vert$ -\/-\/-\/-\/-\/--- $\vert$ -\/-\/-\/-\/-\/-\/-\/-\/-\/-\/-\/--- $\vert$ \tabulinesep=1mm
\begin{longtabu} spread 0pt [c]{*{4}{|X[-1]}|}
\hline
\rowcolor{\tableheadbgcolor}\textbf{ Бит инцидетности }&\textbf{ id вершины }&\textbf{ id ребра }&\textbf{ }\\\cline{1-4}
\endfirsthead
\hline
\endfoot
\hline
\rowcolor{\tableheadbgcolor}\textbf{ Бит инцидетности }&\textbf{ id вершины }&\textbf{ id ребра }&\textbf{ }\\\cline{1-4}
\endhead
0 &id вершины &0..0 &Данные вершины \\\cline{1-4}
\end{longtabu}
```

Структура хранит вершины и исходящие из них ребра. По этой структуре происходит поиск смежных ребер для указанной вершины. Старший бит ключа ({\ttfamily Бит инцидетности}) показывает является ли данное ребро входящим или исходящим (0 -\/ исходящее, 1 -\/ входящее), второй аргумент ключа является идентификатор вершины, а младший аргумент -\/ идентификатор смежного ребра. Данные вершины хранится в значении для записи с ключом, у которого аргумент {\ttfamily id вершины} соответствует этой вершине и аргумент {\ttfamily id ребра} равен 0.
\item {\bfseries Связь ребро -\/$>$ вершина.}

``` $\vert$ Ключ $\vert$ Значение $\vert$ $\vert$ -\/-\/-\/-\/-\/-\/-\/-\/-\/-\/-\/-\/-\/--- $\vert$ -\/-\/-\/-\/-\/--- $\vert$ -\/-\/-\/-\/-\/-\/-\/--- $\vert$ -\/-\/-\/-\/-\/-\/--- $\vert$ \tabulinesep=1mm
\begin{longtabu} spread 0pt [c]{*{4}{|X[-1]}|}
\hline
\rowcolor{\tableheadbgcolor}\textbf{ Бит инцидетности }&\textbf{ id ребра }&\textbf{ id вершины }&\textbf{ }\\\cline{1-4}
\endfirsthead
\hline
\endfoot
\hline
\rowcolor{\tableheadbgcolor}\textbf{ Бит инцидетности }&\textbf{ id ребра }&\textbf{ id вершины }&\textbf{ }\\\cline{1-4}
\endhead
0 &id ребра &0..0 &Вес ребра \\\cline{1-4}
\end{longtabu}
```

Структура хранит ребра и вершины, в которые входят эти ребра. По этой структуре происходит поиск смежных вершин для указанного ребра. Старший бит ключа ({\ttfamily Бит инцидетности}) показывает является ли данное ребро входящим или исходящим (0 -\/ исходящее, 1 -\/ входящее), второй аргумент ключа является идентификатор ребра, а младший аргумент -\/ идентификатор смежной вершины.

Для взвешанных графов в значении для записи с ключом, у которого аргумент {\ttfamily id ребра} соответствует указанному ребру, а аргумент {\ttfamily id вершины} равен 0, значение будет соответствовать весу ребра. Таким образом, модель будет обеспечивать быстрое изменение веса ребра, однако, чтобы получить ребро с минимальны весом, придется вытащить все ребра и найти среди них ребро с минимальным весом.
\end{DoxyEnumerate}

{\bfseries Ограничения\+:} Идентификаторы вершины и ребра не должны быть равны 0.\hypertarget{md_docs__xD0_x93_xD1_x80_xD0_xB0_xD1_x84_xD0_xBE_xD0_xB2_xD1_x8B_xD0_xB5__xD0_xBC_xD0_xBE_xD0_xB4_xD0_xB5_xD0_xBB_xD0_xB8_autotoc_md10}{}\subsection{7 Модель для взвешанного ультраграфа}\label{md_docs__xD0_x93_xD1_x80_xD0_xB0_xD1_x84_xD0_xBE_xD0_xB2_xD1_x8B_xD0_xB5__xD0_xBC_xD0_xBE_xD0_xB4_xD0_xB5_xD0_xBB_xD0_xB8_autotoc_md10}
Данная модель подходит для взвешанных ультрарафов и использует 2 структуры S\+PU\+:


\begin{DoxyEnumerate}
\item {\bfseries Связь вершина -\/$>$ ребро.}

``` $\vert$ Ключ $\vert$ Значение $\vert$ $\vert$ -\/-\/-\/-\/-\/-\/-\/-\/-\/-\/-\/-\/-\/--- + -\/-\/-\/-\/-\/-\/-\/--- + -\/-\/-\/-\/-\/-\/--- + -\/-\/-\/-\/-\/--- + -\/-\/-\/-\/-\/-\/-\/-\/-\/-\/-\/-\/-\/-\/-\/-\/-\/-\/-\/--- $\vert$ $\vert$ Бит инцидетности $\vert$ id вершины $\vert$ Вес ребра $\vert$ id ребра $\vert$ $\vert$ $\vert$ -\/-\/-\/-\/-\/-\/-\/-\/-\/-\/-\/-\/-\/--- + -\/-\/-\/-\/-\/-\/-\/--- + -\/-\/-\/-\/-\/-\/--- + -\/-\/-\/-\/-\/--- + -\/-\/-\/-\/-\/-\/-\/-\/-\/-\/-\/-\/-\/-\/-\/-\/-\/-\/-\/--- $\vert$ $\vert$ 0 $\vert$ id вершины $\vert$ 0..0 $\vert$ Данные вершины $\vert$ $\vert$ 0 $\vert$ id вершины $\vert$ 1..1 $\vert$ Кол-\/во исходящих ребер $\vert$ $\vert$ 1 $\vert$ id вершины $\vert$ 1..1 $\vert$ Кол-\/во входящих ребер $\vert$ $\vert$ 0 $\vert$ 0..0 $\vert$ Общее кол-\/во вершин $\vert$ ```

Структура хранит вершины и исходящие из них ребра. По этой структуре происходит поиск смежных ребер для указанной вершины. Старший бит ключа ({\ttfamily Бит инцидетности}) показывает является ли данное ребро входящим или исходящим (0 -\/ исходящее, 1 -\/ входящее), второй аргумент ключа является идентификатор вершины, третий -\/ вес ребра, а младший аргумент -\/ идентификатор смежного ребра. Данные вершины хранится в значении для записи с ключом, у которого аргумент {\ttfamily id вершины} соответствует этой вершине и аргумент {\ttfamily id ребра} равен 0.

Чтобы не пересчитывать количество смежных ребер предлагается для записи с битом инцидетности равным 0 и старшими аргументами ({\ttfamily Вес ребра} и {\ttfamily id ребра}), у которых биты равны 1, хранить количество входящих ребер, а для записи с битом инцидетности равным 1 хранить количество исходящих ребер.

По адресу 0 хранится общее количество вершин.
\item {\bfseries Связь ребро -\/$>$ вершина.}

``` $\vert$ Ключ $\vert$ Значение $\vert$ $\vert$ -\/-\/-\/-\/-\/-\/-\/-\/-\/-\/-\/-\/-\/--- + -\/-\/-\/-\/-\/--- + -\/-\/-\/-\/-\/-\/-\/--- + -\/-\/-\/-\/-\/-\/-\/-\/-\/-\/-\/-\/-\/-\/-\/-\/-\/-\/-\/-\/-\/-\/-\/-\/-\/-\/-\/-\/-\/-\/-\/-\/-\/--- $\vert$ $\vert$ Бит инцидетности $\vert$ id ребра $\vert$ id вершины $\vert$ $\vert$ $\vert$ -\/-\/-\/-\/-\/-\/-\/-\/-\/-\/-\/-\/-\/--- + -\/-\/-\/-\/-\/--- + -\/-\/-\/-\/-\/-\/-\/--- + -\/-\/-\/-\/-\/-\/-\/-\/-\/-\/-\/-\/-\/-\/-\/-\/-\/-\/-\/-\/-\/-\/-\/-\/-\/-\/-\/-\/-\/-\/-\/-\/-\/--- $\vert$ $\vert$ 0 $\vert$ id ребра $\vert$ 0..0 $\vert$ Вес ребра $\vert$ $\vert$ 0 $\vert$ id ребра $\vert$ 1..1 $\vert$ Кол-\/во вершин, из кот. выходит ребро $\vert$ $\vert$ 1 $\vert$ id ребра $\vert$ 1..1 $\vert$ Кол-\/во вершин, в кот. входит ребро $\vert$ $\vert$ 0 $\vert$ 0..0 $\vert$ Общее количество ребер $\vert$ ```

Структура хранит ребра и вершины, в которые входят эти ребра. По этой структуре происходит поиск смежных вершин для указанного ребра. Старший бит ключа ({\ttfamily Бит инцидетности}) показывает является ли данное ребро входящим или исходящим (0 -\/ исходящее, 1 -\/ входящее), второй аргумент ключа является идентификатор ребра, а младший аргумент -\/ идентификатор смежной вершины.

Для взвешанных графов в значении для записи с ключом, у которого аргумент {\ttfamily id ребра} соответствует указанному ребру, а аргумент {\ttfamily id вершины} равен 0, значение будет соответствовать весу ребра.

Чтобы не пересчитывать количество смежных вершин предлагается для записи с битом инцидетности равным 0 и старшими аргументами ({\ttfamily Вес ребра} и {\ttfamily id ребра}), у которых биты равны 1, хранить количество вершин, из которых выходит ребро, а для записи с битом инцидетности равным 1 хранить количество вершин, в которое входит ребро.

По адресу 0 хранится общее количество ребер.
\end{DoxyEnumerate}

Отличие данной модели от предыдущей -\/ в первую S\+PU структуру был добавлен аргумент {\ttfamily Вес ребра}, благодаря которому, будет осуществлятся быстрый поиск ребер вершины с минимальным / максимальным весом. Однако для изменения веса у ребра придется пробежаться по всем связям {\itshape вершина -\/$>$ ребро} и у каждой изменить вес.

{\bfseries Ограничения\+:}
\begin{DoxyEnumerate}
\item Идентификаторы вершины и ребра не должны быть равны 0, а также все биты идентификаторов не должны быть равными 1.
\item Вес ребра должен быть целым положительным числом.
\end{DoxyEnumerate}

Эта графовая модель для системы с набором дискретной математики наиболее универсальна и будет выбрана в качестве реализации в библиотеки элементов программного интерфейса для обработки графов.\hypertarget{md_docs__xD0_x93_xD1_x80_xD0_xB0_xD1_x84_xD0_xBE_xD0_xB2_xD1_x8B_xD0_xB5__xD0_xBC_xD0_xBE_xD0_xB4_xD0_xB5_xD0_xBB_xD0_xB8_autotoc_md11}{}\subsection{8 Модель ультраграфа с атрибутами для ребер}\label{md_docs__xD0_x93_xD1_x80_xD0_xB0_xD1_x84_xD0_xBE_xD0_xB2_xD1_x8B_xD0_xB5__xD0_xBC_xD0_xBE_xD0_xB4_xD0_xB5_xD0_xBB_xD0_xB8_autotoc_md11}
Данная модель предназначена для ультрарафов, у которых ребры могут иметь несколько атрибутов, по которым должен происходить быстрый доступ для смежных вершин. У такой модели будет достаточно долгое изменение атрибутов ребер, однако, появляется выигрыш при поиске смежных ребер с нужным атрибутом. Ниже показана S\+PU структуры такой модели\+:


\begin{DoxyEnumerate}
\item {\bfseries Связь вершина -\/$>$ ребро.}

``` $\vert$ Ключ $\vert$ Значение $\vert$ $\vert$ -\/-\/-\/-\/-\/-\/-\/-\/-\/-\/-\/-\/-\/--- + -\/-\/-\/-\/-\/-\/-\/--- + -\/-\/-\/-\/-\/-\/-\/-\/--- + -\/-\/-\/-\/-\/-\/-\/-\/-\/--- + -\/-\/-\/-\/-\/--- + -\/-\/-\/-\/-\/-\/-\/-\/-\/-\/-\/-\/-\/-\/-\/-\/-\/-\/-\/--- $\vert$ $\vert$ Бит инцидетности $\vert$ id вершины $\vert$ id атрибута $\vert$ зн. атрибута $\vert$ id ребра $\vert$ $\vert$ $\vert$ -\/-\/-\/-\/-\/-\/-\/-\/-\/-\/-\/-\/-\/--- + -\/-\/-\/-\/-\/-\/-\/--- + -\/-\/-\/-\/-\/-\/-\/-\/--- + -\/-\/-\/-\/-\/-\/-\/-\/-\/--- + -\/-\/-\/-\/-\/--- + -\/-\/-\/-\/-\/-\/-\/-\/-\/-\/-\/-\/-\/-\/-\/-\/-\/-\/-\/--- $\vert$ $\vert$ Бит инцидетности $\vert$ id вершины $\vert$ 0..0 $\vert$ id ребра $\vert$ $\vert$ $\vert$ 0 $\vert$ id вершины $\vert$ 0..0 $\vert$ Данные вершины $\vert$ $\vert$ 0 $\vert$ id вершины $\vert$ 1..1 $\vert$ Кол-\/во исходящих ребер $\vert$ $\vert$ 1 $\vert$ id вершины $\vert$ 1..1 $\vert$ Кол-\/во входящих ребер $\vert$ $\vert$ 0 $\vert$ 0..0 $\vert$ Общее кол-\/во вершин $\vert$ ```
\item {\bfseries Связь ребро -\/$>$ вершина.}

``` $\vert$ Ключ $\vert$ Значение $\vert$ $\vert$ -\/-\/-\/-\/-\/-\/-\/-\/-\/-\/-\/-\/-\/--- + -\/-\/-\/-\/-\/--- + -\/-\/-\/-\/-\/-\/-\/--- + -\/-\/-\/-\/-\/-\/-\/-\/--- + -\/-\/-\/-\/-\/-\/-\/-\/-\/-\/-\/-\/-\/-\/-\/-\/-\/-\/-\/-\/-\/-\/-\/-\/-\/-\/-\/-\/-\/-\/-\/-\/-\/--- $\vert$ $\vert$ Бит инцидетности $\vert$ id ребра $\vert$ id вершины $\vert$ id атрибута $\vert$ $\vert$ $\vert$ -\/-\/-\/-\/-\/-\/-\/-\/-\/-\/-\/-\/-\/--- + -\/-\/-\/-\/-\/--- + -\/-\/-\/-\/-\/-\/-\/--- + -\/-\/-\/-\/-\/-\/-\/-\/--- + -\/-\/-\/-\/-\/-\/-\/-\/-\/-\/-\/-\/-\/-\/-\/-\/-\/-\/-\/-\/-\/-\/-\/-\/-\/-\/-\/-\/-\/-\/-\/-\/-\/--- $\vert$ $\vert$ Бит инцидетности $\vert$ id ребра $\vert$ id вершины $\vert$ 0..0 $\vert$ $\vert$ $\vert$ 0 $\vert$ id ребра $\vert$ 0..0 $\vert$ id атрибута $\vert$ Значение атрибута $\vert$ $\vert$ 0 $\vert$ id ребра $\vert$ 1..1 $\vert$ Кол-\/во вершин, из кот. выходит ребро $\vert$ $\vert$ 1 $\vert$ id ребра $\vert$ 1..1 $\vert$ Кол-\/во вершин, в кот. входит ребро $\vert$ $\vert$ 0 $\vert$ 0..0 $\vert$ Общее количество ребер $\vert$ ```
\end{DoxyEnumerate}

{\bfseries Ограничения\+:}
\begin{DoxyEnumerate}
\item Идентификаторы вершины и ребра не должны быть равны 0, а также все биты идентификаторов не должны быть равными 1.
\item Идентификатор атрибут должен быть натуральным числом.
\item Значение атрибута должно быть целым положительным числом.
\end{DoxyEnumerate}\hypertarget{md_docs__xD0_x93_xD1_x80_xD0_xB0_xD1_x84_xD0_xBE_xD0_xB2_xD1_x8B_xD0_xB5__xD0_xBC_xD0_xBE_xD0_xB4_xD0_xB5_xD0_xBB_xD0_xB8_autotoc_md12}{}\subsection{9 Модель для множества графов}\label{md_docs__xD0_x93_xD1_x80_xD0_xB0_xD1_x84_xD0_xBE_xD0_xB2_xD1_x8B_xD0_xB5__xD0_xBC_xD0_xBE_xD0_xB4_xD0_xB5_xD0_xBB_xD0_xB8_autotoc_md12}
Для реализации нескольких графов на одной системы с набором дискретной математики предлагается в графовые модели добавлять старший аргумент -\/ идентификатор графа. Так, например, будет выглядеть модель для реализации нескольких графов на основе модели взвешанного ультраграфа\+:


\begin{DoxyEnumerate}
\item {\bfseries Структура связь вершина -\/$>$ ребро.}

``` $\vert$ Ключ $\vert$ Значение $\vert$ $\vert$ -\/-\/-\/-\/-\/--- + -\/-\/-\/-\/-\/-\/-\/-\/-\/-\/-\/-\/-\/--- + -\/-\/-\/-\/-\/-\/-\/--- + -\/-\/-\/-\/-\/-\/--- + -\/-\/-\/-\/-\/--- + -\/-\/-\/-\/-\/-\/-\/-\/-\/-\/-\/-\/-\/-\/-\/-\/-\/-\/-\/--- $\vert$ $\vert$ id графа $\vert$ Бит инцидетности $\vert$ id вершины $\vert$ Вес ребра $\vert$ id ребра $\vert$ $\vert$ $\vert$ -\/-\/-\/-\/-\/--- + -\/-\/-\/-\/-\/-\/-\/-\/-\/-\/-\/-\/-\/--- + -\/-\/-\/-\/-\/-\/-\/--- + -\/-\/-\/-\/-\/-\/--- + -\/-\/-\/-\/-\/--- + -\/-\/-\/-\/-\/-\/-\/-\/-\/-\/-\/-\/-\/-\/-\/-\/-\/-\/-\/--- $\vert$ $\vert$ id графа $\vert$ 0 $\vert$ id вершины $\vert$ 0..0 $\vert$ Данные вершины $\vert$ $\vert$ id графа $\vert$ 0 $\vert$ id вершины $\vert$ 1..1 $\vert$ Кол-\/во исходящих ребер $\vert$ $\vert$ id графа $\vert$ 1 $\vert$ id вершины $\vert$ 1..1 $\vert$ Кол-\/во входящих ребер $\vert$ $\vert$ id графа $\vert$ 0 $\vert$ 0..0 $\vert$ Общее кол-\/во вершин $\vert$ ```
\item {\bfseries Структура связь ребро -\/$>$ вершина.}

``` $\vert$ Ключ $\vert$ Значение $\vert$ $\vert$ -\/-\/-\/-\/-\/--- + -\/-\/-\/-\/-\/-\/-\/-\/-\/-\/-\/-\/-\/--- + -\/-\/-\/-\/-\/--- + -\/-\/-\/-\/-\/-\/-\/--- + -\/-\/-\/-\/-\/-\/-\/-\/-\/-\/-\/-\/-\/-\/-\/-\/-\/-\/-\/-\/-\/-\/-\/-\/-\/-\/-\/-\/-\/-\/-\/-\/-\/--- $\vert$ $\vert$ id графа $\vert$ Бит инцидетности $\vert$ id ребра $\vert$ id вершины $\vert$ $\vert$ $\vert$ -\/-\/-\/-\/-\/--- + -\/-\/-\/-\/-\/-\/-\/-\/-\/-\/-\/-\/-\/--- + -\/-\/-\/-\/-\/--- + -\/-\/-\/-\/-\/-\/-\/--- + -\/-\/-\/-\/-\/-\/-\/-\/-\/-\/-\/-\/-\/-\/-\/-\/-\/-\/-\/-\/-\/-\/-\/-\/-\/-\/-\/-\/-\/-\/-\/-\/-\/--- $\vert$ $\vert$ id графа $\vert$ 0 $\vert$ id ребра $\vert$ 0..0 $\vert$ Вес ребра $\vert$ $\vert$ id графа $\vert$ 0 $\vert$ id ребра $\vert$ 1..1 $\vert$ Кол-\/во вершин, из кот. выходит ребро $\vert$ $\vert$ id графа $\vert$ 1 $\vert$ id ребра $\vert$ 1..1 $\vert$ Кол-\/во вершин, в кот. входит ребро $\vert$ $\vert$ id графа $\vert$ 0 $\vert$ 0..0 $\vert$ Общее количество ребер $\vert$ ```
\end{DoxyEnumerate}\hypertarget{md_docs__xD0_x93_xD1_x80_xD0_xB0_xD1_x84_xD0_xBE_xD0_xB2_xD1_x8B_xD0_xB5__xD0_xBC_xD0_xBE_xD0_xB4_xD0_xB5_xD0_xBB_xD0_xB8_autotoc_md13}{}\section{ЗАКЛЮЧЕНИЕ}\label{md_docs__xD0_x93_xD1_x80_xD0_xB0_xD1_x84_xD0_xBE_xD0_xB2_xD1_x8B_xD0_xB5__xD0_xBC_xD0_xBE_xD0_xB4_xD0_xB5_xD0_xBB_xD0_xB8_autotoc_md13}
В результате проеделанной работы были разработаны модели представления различных типов графов для систем с дискретным набором команд. Наиболее перспективная модель в качестве реализации является модель для взвешанного ультрографа, т.\+к. она является наиболее универсальная, с помощью её можно построить любой граф. А также в данной модели наиболее эффективный поиск ребер и вершин. Но в расплату за хороший поиск минимальных и максимальных ребер приходится расплачиваться относительно долгим изменением веса у ребер. 